Formula SAE is a collegiate engineering design competition where teams are tasked to design, manufacture, and compete with an Formula-style race car each year.
Teams are judged on the business aspect, the design aspect, and the performance aspect of their car.
An integral part of the design and performance aspect is real-world validation, for justifying and validating design choices as well as quantifying what changes to the car can provide the best performance.
\begin{figure}[H]
    \centering
    \includegraphics[width=5in]{images/sdm23-unveil.JPG}
    \caption{SDM-23 during unveil testing at Podium Club in Attesa}
    \label{fig:sdm23-unveil}
\end{figure}

The Data Acquisition subteam of Sun Devil Motorsports, Arizona State's Formula SAE team, is responsible for producing a DAQ package that can reliably collect data.
This report will detail the design and implementation of the DAQ package for SDM-23, Sun Devil Motorsports' 2022-23 car, as well as results of implementation and testing and recommendations for future work.
\vspace{1em}

A DAQ package can be broken up into a few high-level components:
\begin{itemize}
    \item Sensors which respond to physical stimuli and either transmit signals or change electrical property (e.g., resistance)
    \item A data logger that can measure electrical signals from sensors, convert it into a useful measurement, and store it in memory
    \item Software, typically on a PC, that can retrieve and process data from the data logger
\end{itemize}
The first two components make up the DAQ electrical system.
The third comes into use when retrieving and processing data from the electrical system.
\section{Project Goals}
The goals for SDM-23's Data Acquisition package are to:
\begin{itemize}
    \item Build upon SDM-22's DAQ package concept
    \item Provide consistent and useful data
    \item Increase the reliability and professionalism of DAQ wiring
\end{itemize}

\section{Project Requirements}
The following are the project requirements for the Data Acquisition package.
System-level requirements were determined by the DAQ subteam, and sensor requirements and priorities were given by other subteam leads.
\begin{enumerate}
    \item System-Level Requirements
    \begin{enumerate}
        \item The DAQ package must be standalone; the car's operation shall not need to depend on the functionality of any DAQ component.
        \item The DAQ electrical system shall be powered off of 12V from the ECU.
        \item The DAQ electrical system shall allow for expansion, i.e. additional sensors or modules can be plugged into it without major changes.
        \item Data retrieval shall be able to be accomplished by plugging a computer into the system using USB.
        \item Any communication between microcontrollers on the car shall use CAN.
    \end{enumerate}
    \item Sensor Requirements
    \begin{enumerate}
        \item Aerodynamics
        \begin{enumerate}
            \item Strain gauges on push/pull rods
            \item air speed measurements
        \end{enumerate}
        \item Brakes
        \begin{enumerate}
            \item Brake pressure
            \item Brake rotor temperature
            \item Brake fluid temperature
        \end{enumerate}
        \item Drivetrain
        \begin{enumerate}
            \item Strain gauges on drivetrain components
        \end{enumerate}
        \item Suspension
        \begin{enumerate}
            \item 3 axis gyroscope and accelerometer
            \item Multiple accelerometers
            \item Linear potentiometers
        \end{enumerate}
        \item Systems
        \begin{enumerate}
            \item Steering wheel angle
            \item Steering wheel force
        \end{enumerate}
    \end{enumerate}
\end{enumerate}

